\documentclass[12pt]{article}
\usepackage[utf8]{inputenc}
\usepackage[italian]{babel}
\usepackage{graphicx}
\usepackage{xcolor}
\usepackage[a4paper,top=15mm, bottom=15mm, left=15mm, right=15mm]{geometry}

\begin{document}
	\title{Progetto ristrutturazione sito}
	\maketitle
	
	\begin{center}
		\textcolor{red}{\huge{Il Baronio ha bisogno di voi!}}
	\end{center}
	Il sito della scuola purtroppo è ormai diventato antiquato e c'è bisogno di un bel restyling. Purtroppo il tecnico informatico che sarebbe adibito al suo svecchiamento è andato in pensione, per cui non è rimasto alcuno a gestirlo.\\
	Servono idee nuove ed innovative, oltre ad un buon occhio, per la scelta del design del nuovo sito e la sua implementazione. E chi meglio di te, che sei una mente giovane e fresca, può portare avanti questo processo? Sei degno di questa sfida?\\
	\section{Prerequisiti}
	\begin{itemize}
		\item Account Google personale
		\item Postazione del PC
		\item Utilizzo di Google Sites (sites.google.com)
	\end{itemize}
	\section{Regole}
	\begin{itemize}
		\item Il lavoro deve essere completamente personale
		\item La valutazione avverrà tramite classifica, ossia verrà valutato il vostro sito da una giuria di professori e il voto andrà a scalare (ovviamente, se il progetto non è completo o adeguato alle specifiche verrà valutato negativo)
		\item Potete lavorare anche da casa, basta collegarsi al sito col proprio account personale
		\item Aggiungete il professore come collaboratore del sito, in modo che possa darvi dritte su come procedere
	\end{itemize}
	\section{Obiettivo}
	L'obiettivo è, se non avete capito, quello di rimodellare il sito del Baronio con uno stile moderno e piacevole agli occhi. Non occorre che mettiate tutti i contenuti secondari, ma il sito deve essere completo delle informazioni necessarie alla presentazione della scuola, con tutti i link fondamentali ed i testi nelle giuste sezioni (casomai fate copia e incolla per le parti lunghe). Scegliete la struttura e l'impostazione che più preferite, ma seguite le regole dette in classe nella parte di presentazione dei siti web, in modo da fare un lavoro efficace.
	
	\begin{center}
		\textcolor{blue}{\huge{GOOD LUCK!}}
	\end{center}
	
\end{document}