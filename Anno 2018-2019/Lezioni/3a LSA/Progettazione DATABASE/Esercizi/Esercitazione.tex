\documentclass{article}
\usepackage[utf8]{inputenc}
\usepackage[italian]{babel}
\usepackage{graphicx}
\usepackage{xcolor}
\usepackage[a4paper,top=15mm, bottom=15mm, left=15mm, right=15mm]{geometry}

\begin{document}
	
\begin{center}
	\fbox{\fbox{\parbox{5.5in}{\centering Esercitazione Diagramma ER e Schema Logico}}}
\end{center}

Dei seguenti esercizi, progetta il corrispettivo diagramma ER e fai la traduzione in schema logico.

\begin{enumerate}
	\item Un istituto di ricerca è composto di sezioni. Una sezione è identificata da un codice; di una sezione interessa il nome, il responsabile, i ricercatori che vi afferiscono. Un ricercatore è identificato da un codice; di un ricercatore interessa il nome, la sezione di appartenenza, i progetti cui partecipa. Un progetto è identificato da un codice; di un progetto interessa l'obiettivo, il responsabile, i ricercatori che vi partecipano.
	\item Si vuole creare un gestionale di una rete di biblioteche. Ogni biblioteca ha un nome e una città in cui è collocata, oltre che l'indirizzo specifico all'interno della città. Al loro interno, ogni biblioteca è divisa in reparti, i quali hanno un nome ed un identificativo, oltre che un collocamento nella biblioteca (primo piano, ala est, etc.). Ogni reparto contiene dei libri, dei quali si vuole salvare il titolo, il codice ISBN e l'autore. I libri possono essere presi a prestito dagli utenti, a patto che essi siano registrati con una tessera, un nome ed un cognome. I prestiti vengono salvati con la data di inizio e la data di scadenza.
	\item Si vuole creare il sistema di gestione di un nuovo programma di chat. Esso ha l'obiettivo di far comunicare le persone, le quali si registrano con nome, cognome, numero di telefono e mail e, ad ognuno, viene fornito un codice identificativo personale di accesso. Le persone possono scambiarsi messaggi, i quali vengono registrati con contenuto, data di invio, persona a cui vengono mandati, persona che li invia. Una chat è un insieme di messaggi e viene identificata da un codice particolare. La rubrica del sistema di chat contiene le persone con cui la persona ha iniziato almeno una chat. Infine, c'è la possibilità di creare dei gruppi, i quali hanno un nome, delle persone che lo compongono e dei messaggi all'interno.
	\item Si vuole creare il database di un campionato di calcio. Esso ha delle squadre iscritte tramite nome, anno di fondazione, capitale sociale e logo. Ogni squadra gioca in uno stadio, che possiede un nome, un indirizzo ed una capienza di persone massime. I tifosi possono andare allo stadio a patto che siano registrati attraverso la tessera della società, in cui si salvano il codice fiscale, il nome, il cognome e la data di nascita della persona. Ogni squadra ha sotto contrattto un determinato numero di giocatori, identificata da un codice, un nome, un cognome, la data di nascita, il paese di nascita e lo stipendio. Ogni calciatore possiede un cartellino, identificato da un codice e da un valore commerciale (indicativo del prezzo per acquistare il giocatore). All'interno del campionato, vengono svolte delle partite tra le varie squadra e vengono registrate con data dell'incontro ed esito.
\end{enumerate}
\begin{center}
	\fbox{\fbox{\parbox{5.5in}{\centering Esercitazione Diagramma ER e Schema Logico}}}
\end{center}

Dei seguenti esercizi, progetta il corrispettivo diagramma ER e fai la traduzione in schema logico.
\begin{enumerate}
	\item Un istituto di ricerca è composto di sezioni. Una sezione è identificata da un codice; di una sezione interessa il nome, il responsabile, i ricercatori che vi afferiscono. Un ricercatore è identificato da un codice; di un ricercatore interessa il nome, la sezione di appartenenza, i progetti cui partecipa. Un progetto è identificato da un codice; di un progetto interessa l'obiettivo, il responsabile, i ricercatori che vi partecipano.
	\item Si vuole creare un gestionale di una rete di biblioteche. Ogni biblioteca ha un nome e una città in cui è collocata, oltre che l'indirizzo specifico all'interno della città. Al loro interno, ogni biblioteca è divisa in reparti, i quali hanno un nome ed un identificativo, oltre che un collocamento nella biblioteca (primo piano, ala est, etc.). Ogni reparto contiene dei libri, dei quali si vuole salvare il titolo, il codice ISBN e l'autore. I libri possono essere presi a prestito dagli utenti, a patto che essi siano registrati con una tessera, un nome ed un cognome. I prestiti vengono salvati con la data di inizio e la data di scadenza.
	\item Si vuole creare il sistema di gestione di un nuovo programma di chat. Esso ha l'obiettivo di far comunicare le persone, le quali si registrano con nome, cognome, numero di telefono e mail e, ad ognuno, viene fornito un codice identificativo personale di accesso. Le persone possono scambiarsi messaggi, i quali vengono registrati con contenuto, data di invio, persona a cui vengono mandati, persona che li invia. Una chat è un insieme di messaggi e viene identificata da un codice particolare. La rubrica del sistema di chat contiene le persone con cui la persona ha iniziato almeno una chat. Infine, c'è la possibilità di creare dei gruppi, i quali hanno un nome, delle persone che lo compongono e dei messaggi all'interno.
	\item Si vuole creare il database di un campionato di calcio. Esso ha delle squadre iscritte tramite nome, anno di fondazione, capitale sociale e logo. Ogni squadra gioca in uno stadio, che possiede un nome, un indirizzo ed una capienza di persone massime. I tifosi possono andare allo stadio a patto che siano registrati attraverso la tessera della società, in cui si salvano il codice fiscale, il nome, il cognome e la data di nascita della persona. Ogni squadra ha sotto contrattto un determinato numero di giocatori, identificata da un codice, un nome, un cognome, la data di nascita, il paese di nascita e lo stipendio. Ogni calciatore possiede un cartellino, identificato da un codice e da un valore commerciale (indicativo del prezzo per acquistare il giocatore). All'interno del campionato, vengono svolte delle partite tra le varie squadra e vengono registrate con data dell'incontro ed esito.
\end{enumerate}
	
\end{document}