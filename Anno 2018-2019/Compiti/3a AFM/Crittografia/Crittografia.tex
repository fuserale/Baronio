\documentclass[addpoints]{exam}
\usepackage[utf8]{inputenc}
\usepackage{multicol}
 
\begin{document}
	
\begin{center}
	\fbox{\fbox{\parbox{5.5in}{\centering
	\huge{COMPITO DI CRITTOGRAFIA}}}}
\end{center}
 
\vspace{5mm}
 
\makebox[\textwidth]{Nome e Cognome:\enspace\hrulefill}
 
\vspace{5mm}
 
\makebox[\textwidth]{Data e Classe:\enspace\hrulefill}
 
 
\vspace{10mm}

Risolvi i seguenti problemi, mostrando tutti i passaggi necessari alla loro risoluzione e spiegando eventuali scelte personali.
\begin{questions}
	
\question[1] Spiega cos'è la crittografia e perché viene utilizzata nel mondo aziendale
	
\question[2] Applicando l'algoritmo del cifrario di Cesare a 26 lettere, cripta il seguente messaggio, usando come chiave K=7: \textbf{SEGAFREDO, EL SE QUERZE}

\question[2] Applicando l'algoritmo del cifrario di Cesare a 26 lettere, decripta il seguente messaggio, sapendo che è stata usata una chiave K=4 per cifrarlo: \textbf{PI QSWGLI ZSPERS RIP GMIPS}

\question[2] Usando l'algoritmo del cifrario a trasposizione, cripta il seguente messaggio, usando come chiave k=DIECI: \textbf{ATTACCATE IL NEMICO SUL FIANCO SINISTRO}

\question[3] Spiega come funziona la crittografia a chiave asimmetrica (esempio di Alice e Bob), aiutandoti se necessario con dei disegni

\end{questions}

\vspace{10mm}

\begin{center}
	\gradetable[h][questions]
\end{center}

\end{document}