\documentclass[addpoints]{exam}
\usepackage[utf8]{inputenc}
\usepackage{multicol}
\usepackage[a4paper,top=15mm, bottom=15mm, left=15mm, right=15mm]{geometry}
 
\begin{document}
 
\begin{center}
	\fbox{\fbox{\parbox{5.5in}{\centering Istituti Card. C. Baronio - Vicenza \\ Anno scolastico 2018/2019 \\ Compito di Informatica \\ HTML \& CSS}}}
\end{center} 
 
\vspace{5mm}
 
\makebox[\textwidth]{Nome e Cognome:\enspace\hrulefill Data e Classe:\enspace\hrulefill}
 
\vspace{5mm}
 
\begin{questions}
\begin{multicols}{2}

\question[\half] HTML è un linguaggio di
\begin{checkboxes}
	\choice Programmazione
	\choice Descrizione
	\choice Sperimentazione
	\choice SubProgrammazione
\end{checkboxes}

\question[\half] Il CSS serve a 
\begin{checkboxes}
	\choice Inserire contenuti
	\choice Inserire immagini
	\choice Abbellire la pagina
	\choice Controllare la pagina
\end{checkboxes}

\question[\half] Il tag per il link è 
\begin{checkboxes}
	\choice a
	\choice p
	\choice br
	\choice hr
\end{checkboxes}

\question[\half] Un attributo è identificato da
\begin{checkboxes}
	\choice Coppia valore-nome
	\choice Coppia chiave-chiave
	\choice Coppia tag-valore
	\choice Coppia nome-valore
\end{checkboxes}

\question[\half] La struttura di una pagina WEB:
\begin{checkboxes}
	\choice head, doctype, body
	\choice head, body, doctype
	\choice body, doctype, head
	\choice doctype, head, body
\end{checkboxes}

\question[\half] Il tag DIV:
\begin{checkboxes}
	\choice E' un contenitore
	\choice Separa contenuti
	\choice Serve per i titoli
	\choice Colora il testo
\end{checkboxes}

\question[\half] Gli attributi del CSS vanno inseriti
\begin{checkboxes}
	\choice Nel tag di apertura
	\choice Nel tag di chiusura
	\choice In entrambi i tag
	\choice All'inizio del documento
\end{checkboxes}

\question[\half] Invio dati in un form:
\begin{checkboxes}
	\choice text
	\choice textarea
	\choice button
	\choice url
\end{checkboxes}

\question[\half] A cosa serve l'HTML?:
\begin{checkboxes}
	\choice Creare slide di presentazione
	\choice Creare applet
	\choice Creare pagine web
	\choice Creare documenti
\end{checkboxes}

\question[\half] Per avere il testo in grassetto uso:
\begin{checkboxes}
	\choice b
	\choice grasset
	\choice underline
	\choice italic
\end{checkboxes}

\question[\half] A cosa servono i commenti?
\begin{checkboxes}
	\choice Migliorare la leggibilità
	\choice Sottilineare il codice
	\choice Spiegare il codice
	\choice Inserire contenuto
\end{checkboxes}

\question[\half] L'ordine di importanza per i titoli è:
\begin{checkboxes}
	\choice h2,h3,h1
	\choice h1.h2,h3
	\choice h3,h2,h1
	\choice non c'è ordine
\end{checkboxes}

\question[\half] L'attributo obbligatorio per il tag image:
\begin{checkboxes}
	\choice src
	\choice img
	\choice alt
	\choice dimension
\end{checkboxes}

\question[\half] L'attributo title:
\begin{checkboxes}
	\choice Inserito nell'head
	\choice Inserito in ogni tag
	\choice Inserito nel body
	\choice Non esiste
\end{checkboxes}

\question[\half] L'attributo class="w3-black":
\begin{checkboxes}
	\choice Colora il testo di nero
	\choice Toglie lo sfondo
	\choice Colora lo sfondo di nero
	\choice Rende tutti gli elementi neri
\end{checkboxes}

\question[\half] L'attributo class="w3-text-white":
\begin{checkboxes}
	\choice Colora il testo di bianco
	\choice Toglie lo sfondo
	\choice Colora lo sfondo di bianco
	\choice Rende tutti gli elementi bianchi
\end{checkboxes}

\question[\half] Per creare una tabella con una righe e due colonne
\begin{checkboxes}
	\choice $<tr><td></td><td></td></tr>$
	\choice $<td><tr></tr><tr></tr></td>$
	\choice $<tr><tr></tr><td></td></tr>$
	\choice $<tr><td></td><td></td></td>$
\end{checkboxes}

\question[\half] Per creare una lista non ordinata
\begin{checkboxes}
	\choice $<li><ul></ul><ul></ul></li>$
	\choice $<ul><li></li><li></li></ul>$
	\choice $<li><ol></ol><ol></ol></li>$
	\choice $<ol><li></li><li></li></ol>$
\end{checkboxes}

\question[\half] Per creare una lista ordinata
\begin{checkboxes}
	\choice $<li><ul></ul><ul></ul></li>$
	\choice $<ul><li></li><li></li></ul>$
	\choice $<li><ol></ol><ol></ol></li>$
	\choice $<ol><li></li><li></li></ol>$
\end{checkboxes}

\question[\half] I tag caratteristici di un form sono:
\begin{checkboxes}
	\choice text, textarea, button, img, tel, url
	\choice text, textarea, button, a, tel, url, table
	\choice text, selection, button, tel, url
	\choice text, textarea, form, select, tel, url
\end{checkboxes}
\end{multicols}
\bonusquestion[2] Cosa visualizza il seguente codice HTML?
\begin{verbatim}
<!DOCTYPE html>
<html>
<head>
<title>Titolo della scheda</title>
<link rel="stylesheet" href="https://www.w3schools.com/w3css/4/w3.css">
</head>

<body>
<h1 class="w3-red w3-text-white">Sono un titolo</h1>
<p class="w3-black w3-text-green">
Sono un paragrafo
</p>
<a href="www.google.it" class="w3-text-red">Link a Google</a>
</body>
</html>
\end{verbatim}

\makeemptybox{2in}

\end{questions}

\begin{center}
	\multirowcombinedgradetable{2}[questions]
\end{center}

\end{document}