\documentclass[addpoints]{exam}
\usepackage[utf8]{inputenc}
\usepackage{multicol}
 
\begin{document}
 
\begin{center}
\fbox{\fbox{\parbox{5.5in}{\centering
Rispondi a 5 tra le seguenti 7 domande. Rispondi alle domande nello spazio predisposto.}}}
\end{center}
 
\vspace{5mm}
 
\makebox[\textwidth]{Nome e Cognome:\enspace\hrulefill}
 
\vspace{5mm}
 
\makebox[\textwidth]{Data e Classe:\enspace\hrulefill}
 
\begin{questions}
	
\question[2] Spiega quali sono i vantaggi di usare una rete condivisa
\fillwithlines{1in}

\question[2] Come funziona l’architettura client/server? Puoi rispondere usando direttamente un esempio
\fillwithlines{2in}
 
\question[2] Descrivi una tipologia di rete (stella, anello, dorsale, maglia) con pregi e difetti, usando anche un’immagine
\fillwithlines{1in}

\question[2] Spiega una tipologia di e-commerce (business-to-business, business-to-consumer, consumer-to-consumer, intra-business)
\fillwithlines{2in}

\question[2] Quali sono vantaggi e svantaggi del cloud computing?
\fillwithlines{1in}

\question[2] Cosa vuol dire per un sistema informatico “essere sicuro”?
\fillwithlines{2in}

\question[2] Qual è, secondo te, un possibile nuovo approccio per la struttura della rete? Spiega i punti di forza e di debolezza della tua tecnologia
\fillwithlines{2in}
\end{questions}

\begin{center}
	\combinedgradetable[h][questions]
\end{center}

\end{document}