\documentclass[addpoints]{exam}
\usepackage[utf8]{inputenc}
\usepackage{multicol}
 
\begin{document}
 
\begin{center}
\fbox{\fbox{\parbox{5.5in}{\centering
\huge{DIAGRAMMI ER}}}}
\end{center}
 
\vspace{5mm}
 
\makebox[\textwidth]{Nome e Cognome:\enspace\hrulefill}
 
\vspace{5mm}
 
\makebox[\textwidth]{Data e Classe:\enspace\hrulefill}
 
\begin{questions}
	
\question[5]Un'università vuole raccogliere ed organizzare le informazioni sui propri studenti in relazione ai corsi che essi frequentano ed agli esami che essi sostengono. Uno studente è rappresentato da un nome, un cognome ed una matricola. Ogni corso viene etichettato con un nome, un professore che la insegna e un numero di CFU. 
\begin{enumerate}
	\item Individuare le entità, gli attributi di ognuna di essa e le associazioni tra di esse
	\item Disegnare il modello E/R
	\item Verificare lo schema con le regole di lettura
\end{enumerate}
 
\question[5] 	Un ospedale è composto da reparti. A un reparto afferiscono (ossia sono presenti) medici. Un paziente può essere ricoverato in un reparto e si tiene traccia del suo codice, nome, cognome, codice fiscale, data e luogo di nascita, sesso, data di ricovero. Di un medico si memorizza codice, nome, cognome, data e luogo di nascita. I medici effettuano visite sui pazienti. I pazienti subiscono le visite. Di una visita si memorizza la data e l’esito. Sui pazienti, inoltre, vengono effettuati esami di laboratorio. Di un esame si memorizza il tipo, la data e l’esito. NB: un medico può visitare un paziente più volte in date diverse, quindi modella le visite come entità
\begin{enumerate}
	\item Individuare le entità, gli attributi di ognuna di essa e le associazioni tra di esse
	\item Disegnare il modello E/R
	\item Verificare lo schema con le regole di lettura
\end{enumerate}

\end{questions}

\begin{center}
	\gradetable[h][questions]
\end{center}

\end{document}