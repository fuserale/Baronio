\documentclass[addpoints]{exam}
\usepackage[utf8]{inputenc}
\usepackage{multicol}
 
\begin{document}
 
\begin{center}
\fbox{\fbox{\parbox{5.5in}{\centering Compito di Informatica \\ Diagrammi ER e schema logico}}}
\end{center}
 
\vspace{5mm}
 
\makebox[\textwidth]{Nome e Cognome:\enspace\hrulefill}
 
\vspace{5mm}
 
\makebox[\textwidth]{Data e Classe:\enspace\hrulefill}
 
\begin{questions}
	
\question[5]Una società di analisi dei consumi vuole controllare gli acquisti fatti dai clienti presso una loro negozio. Ogni utente possiede una tessera personale dove tiene i punti fedeltà. I prodotti acquistati hanno un codice a barre, un nome ed una descrizione. I negozi sono dislocati con indirizzi diversi, che li identificano. Ogni acquisto viene registrato sulla tessera e vengono aggiunti i corrispettivi punti. Ogni negozio ha un proprio listino e può proporre prezzi diversi per gli stessi prodotti.

\begin{enumerate}
	\item Individuare le entità, gli attributi di ognuna di essa e le associazioni tra di esse
	\item Disegnare il modello E/R
	\item Verificare lo schema con le regole di lettura
\end{enumerate}
 
\question[5] 	Una linea di trasporto pubblico è caratterizzata da un numero ed è composta da più fermate. Di ogni fermata si memorizza il nome e l’indirizzo. Teniamo inoltre traccia dei passaggi. Una linea effettua un passaggio a una fermata in determinati orari. Su ogni trasporto possono salire delle persone. Le persone vengono registrate con codice tessera, nome, cognome e data di nascita, più una foto di riconoscimento. Inoltre, si vuole tenere conto di tutte le volte che una persona sale su un autobus, timbrando la propria tessera. Infine, degli autisti si vuole salvare il codice del cartellino, il nome, cognome e data di nascita, oltre che una foto di riconoscimento. Ogni autista, per semplificazione, guida sempre lo stesso autobus. NB: prestate attenzione a non trascurare il fatto che una linea può effettuare più passaggi alla stessa fermata
\begin{enumerate}
	\item Individuare le entità, gli attributi di ognuna di essa e le associazioni tra di esse
	\item Disegnare il modello E/R
	\item Verificare lo schema con le regole di lettura
\end{enumerate}

\end{questions}

\begin{center}
	\gradetable[h][questions]
\end{center}

\end{document}