\documentclass[addpoints]{exam}
\usepackage[utf8]{inputenc}
\usepackage{multicol}
 
\begin{document}
 
\begin{center}
\fbox{\fbox{\parbox{5.5in}{\centering
\huge{DIAGRAMMI ER}}}}
\end{center}
 
\vspace{5mm}
 
\makebox[\textwidth]{Nome e Cognome:\enspace\hrulefill}
 
\vspace{5mm}
 
\makebox[\textwidth]{Data e Classe:\enspace\hrulefill}
 
\begin{questions}
	
\question[5] Si vuole organizzare un sondaggio in merito al lavoro degli impiegati nello svolgimento delle pratiche. Queste vengono individuate tramite un codice ed un argomento da scegliere tra "automobilistica", "previdenziale" e "sanitaria". Il sondaggio vuole tenere conto anche delle città italiane in cui lavorano gli impiegati, contraddistinte da un nome e da un CAP. Degli impiegati vogliamo sapere il nome, il cognome ed il ruolo aziendale.
\begin{enumerate}
	\item Individuare le entità, gli attributi di ognuna di essa e le associazioni tra di esse
	\item Disegnare il modello E/R
	\item Verificare lo schema con le regole di lettura
\end{enumerate}
 
\question[5] 		Una catena di negozi è costituita da un certo numero di centri vendita di cui interessano il codice, la ragione sociale e l’indirizzo. I centri vendita effettuano ordini (caratterizzati da un codice e dalla data d’ordine) che comprendono gli articoli da vendere, i quali appartengono a diverse categorie merceologiche (ad esempio“alimentari”, “abbigliamento” ecc.).
\begin{enumerate}
	\item Individuare le entità, gli attributi di ognuna di essa e le associazioni tra di esse
	\item Disegnare il modello E/R
	\item Verificare lo schema con le regole di lettura
\end{enumerate}

\end{questions}

\begin{center}
	\gradetable[h][questions]
\end{center}

\end{document}