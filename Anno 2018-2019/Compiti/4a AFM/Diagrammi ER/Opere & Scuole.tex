\documentclass[addpoints]{exam}
\usepackage[utf8]{inputenc}
\usepackage{multicol}
 
\begin{document}
 
\begin{center}
\fbox{\fbox{\parbox{5.5in}{\centering
\huge{DIAGRAMMI ER}}}}
\end{center}
 
\vspace{5mm}
 
\makebox[\textwidth]{Nome e Cognome:\enspace\hrulefill}
 
\vspace{5mm}
 
\makebox[\textwidth]{Data e Classe:\enspace\hrulefill}
 
\begin{questions}
	
\question[5] 	Si vuole organizzare un database che archivi le opere d’arte presenti nei musei italiani. Tali opere sono identificate tramite un codice identificativo, il titolo ed il valore commerciale. Il database vuole gestire anche un’anagrafica degli artisti che sono esposti nei musei italiani ed un’anagrafica delle città italiane viste sia come sede dei musei stessi, sia come luogo di nascita degli artisti.
\begin{enumerate}
	\item Individuare le entità, gli attributi di ognuna di essa e le associazioni tra di esse
	\item Disegnare il modello E/R
	\item Verificare lo schema con le regole di lettura
\end{enumerate}
 
\question[5] 		Un’indagine statistica vuole organizzare un database in merito a scuole e docenti (individuati da un codice univoco, dal proprio nominativo e dal proprio indirizzo). Si vuole limitare l’indagine ai soli docenti che insegnano presso i capoluoghi di provincia italiani. Ogni docente, inoltre, è nato in una specifica città italiana (capoluogo o meno che sia). Ogni scuola risiede in uno specifico capoluogo ed è associata ad un grado di istruzione (quali ad esempio “superiore”, “media”,.. ecc.).
\begin{enumerate}
	\item Individuare le entità, gli attributi di ognuna di essa e le associazioni tra di esse
	\item Disegnare il modello E/R
	\item Verificare lo schema con le regole di lettura
\end{enumerate}

\end{questions}

\begin{center}
	\gradetable[h][questions]
\end{center}

\end{document}