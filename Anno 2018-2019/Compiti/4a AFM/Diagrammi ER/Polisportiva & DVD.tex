\documentclass[addpoints]{exam}
\usepackage[utf8]{inputenc}
\usepackage{multicol}
 
\begin{document}
 
\begin{center}
\fbox{\fbox{\parbox{5.5in}{\centering
\huge{DIAGRAMMI ER}}}}
\end{center}
 
\vspace{5mm}
 
\makebox[\textwidth]{Nome e Cognome:\enspace\hrulefill}
 
\vspace{5mm}
 
\makebox[\textwidth]{Data e Classe:\enspace\hrulefill}
 
\begin{questions}
	
\question[5] 	Una società polisportiva vuole organizzare dei corsi tenuti da propri istruttori. Di questi si vuole conoscere il nome, il cognome e le ore svolte nella palestra. Ogni corso è contraddistinto da un nome, da un istruttore che lo tiene e da un orario. Ogni corso è specifico per una disciplina ed è frequentato da soci della società. Di un socio si tiene traccia del nominativo, del numero corsi che tiene ed è identificato dal proprio numero di tessera.
\begin{enumerate}
	\item Individuare le entità, gli attributi di ognuna di essa e le associazioni tra di esse
	\item Disegnare il modello E/R
	\item Verificare lo schema con le regole di lettura
\end{enumerate}
 
\question[5] 		Una società che gestisce un noleggio di film dvd vuole organizzare un database a fini statistici. Ogni noleggio è individuato dal codice e dalla data di noleggio. A tale scopo è interessata a catalogare i suoi clienti tramite il numero di tessera, il nominativo e la data di nascita. Inoltre di ogni dvd sono noti il titolo ed il regista, oltre le informazioni utili allo scopo.
\begin{enumerate}
	\item Individuare le entità, gli attributi di ognuna di essa e le associazioni tra di esse
	\item Disegnare il modello E/R
	\item Verificare lo schema con le regole di lettura
\end{enumerate}

\end{questions}

\begin{center}
	\gradetable[h][questions]
\end{center}

\end{document}