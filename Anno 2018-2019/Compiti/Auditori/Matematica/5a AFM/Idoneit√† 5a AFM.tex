\documentclass[addpoints]{exam}
\usepackage[utf8]{inputenc}
\usepackage{multicol}
\usepackage{graphicx}
\usepackage{amsmath}
\usepackage[a4paper,top=15mm, bottom=15mm, left=15mm, right=15mm]{geometry}
 
\begin{document}
 
\begin{center}
	\fbox{\fbox{\parbox{5.5in}{\centering Istituti Card. C. Baronio - Vicenza \\ Anno scolastico 2018/2019 \\ Compito di Matematica \\ Idoneità 5a AFM}}}
\end{center} 
 
\vspace{5mm}
 
\makebox[\textwidth]{Nome e Cognome:\enspace\hrulefill Data:\enspace\hrulefill}
 
\vspace{5mm}
 

\begin{questions}
	\question[2 \half] Determina il dominio delle seguenti funzioni:
	\begin{align*}
	y=x^2-3x+2 \hspace{15mm} 
	y=\sqrt{2x-2} \hspace{15mm}
	y=\frac{x^2+4x-1}{x+2} \hspace{15mm}
	y=\frac{x+1}{x^2+1}
	\end{align*}
	
	\question[2 \half] Indica se le seguenti funzioni sono pari, dispari o non presentano simmetrie:
	\begin{align*}
	y=2x^3+5x+1 \hspace{15mm}
	y=x^4-2x^2+3 \hspace{15mm}
	y=\frac{x+1}{x^2-1}
	\end{align*}
	
	\question[2 \half] Calcola il valore dei seguenti limiti, risolvendo eventuali forme indeterminate:
	\begin{align*}
	\lim_{x \to -2} \frac{x+1}{x-3} \hspace{8mm}
	\lim_{x \to -3} \frac{x+6}{x+3} \hspace{8mm}
	\lim_{x \to \infty} \frac{x^2}{x} \hspace{8mm}
	\lim_{x \to \infty} \frac{3x-7}{5x+7} \hspace{8mm}
	\lim_{x \to \infty} \frac{x+3}{x^2-7} \hspace{8mm}
	\lim_{x \to -\infty} 6x^2-4x
	\end{align*}
	
	\question[2 \half] Calcola le seguenti derivate e, se necessario, indica il dominio di esistenza:
	\begin{align*}
	D(1344) \hspace{8mm}
	D(45x) \hspace{8mm}
	D(2x^2) \hspace{8mm}
	D(\frac{4}{x}) \hspace{8mm}
	D(x^2+3x^3-2x) \hspace{8mm}
	D((x^2+1)(x^3-2))
	\end{align*}
\end{questions}

\begin{center}
	\gradetable[h][questions]
\end{center}

\end{document}