\documentclass[addpoints]{exam}
\usepackage[utf8]{inputenc}
\usepackage{multicol}
\usepackage{graphicx}
\usepackage{amsmath}
\usepackage[a4paper,top=15mm, bottom=15mm, left=15mm, right=15mm]{geometry}

%\printanswers

\newcommand{\tf}[1][{}]{%
	\fillin[#1][0.25in]%
}
 
\begin{document}
 
\begin{center}
	\fbox{\fbox{\parbox{5.5in}{\centering Istituti Card. C. Baronio - Vicenza \\ Anno scolastico 2018/2019 \\ Compito di Informatica \\ Idoneità 5a LSA}}}
\end{center} 
 
\vspace{5mm}
 
\makebox[\textwidth]{Nome e Cognome:\enspace\hrulefill Data:\enspace\hrulefill}
 
\vspace{5mm}
 
Vicino ad ogni quesito, scrivi se la risposta è vera (T) o falsa (F)
\begin{multicols}{2}
\begin{questions}

\question \tf[T] Il codice sorgente è quello che contiene il testo di un programma

\question \tf[F] Il codice viene prima eseguito e poi compilato

\question \tf[T] Un variabile è il contenitore di un valore

\question \tf[T] In programmazione esistono più tipi di variabili

\question \tf[F] Il tipo double viene usato per i numeri interi

\question \tf[F] Il tipo boolean è usato per le parole

\question \tf[T] Il tipo char è usato per le parole

\question \tf[T] Un metodo ci permette di fare una determinata operazioni su una variabile

\question \tf[T] Per ottenere la lunghezza di una stringa, basta usare il metodo lenght

\question \tf[F] Per ottenere il resto di una divisione, si usa l'operatore *

\question \tf[F] Se divido una variabile intera per un variabile double, ottengo una variabile double

\question \tf[T] Per far inserire dati ad un'utente si usa la classe Scanner

\question \tf[T] Esistono 3 tipi di cicli: for, do e while

\question \tf[F] Per eseguire un controllo su una variabile si usa il costrutto else

\question \tf[T] Il comando else va sempre dopo il comando if

\question \tf[F] Uso il ciclo while quando conosco a priori il numero di volte che ripeterò l'operazione

\question \tf[T] Uso il ciclo for quando conosco a priori il numero di volte che ripeterò l'operazione 

\question \tf[T] Gli array sono contenitori di dati omogenei

\question \tf[F] L'indice di partenza di una array è 1

\question \tf[T] Per scorrere tutti gli elementi di un array devo per forza usare un ciclo


\end{questions}
\end{multicols}

\end{document}