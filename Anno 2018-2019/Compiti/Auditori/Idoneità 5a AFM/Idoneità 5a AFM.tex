\documentclass[addpoints]{exam}
\usepackage[utf8]{inputenc}
\usepackage{multicol}
\usepackage{graphicx}
\usepackage{amsmath}
\usepackage[a4paper,top=15mm, bottom=15mm, left=15mm, right=15mm]{geometry}
 
 
%\printanswers

\newcommand{\tf}[1][{}]{%
	\fillin[#1][0.25in]%
}

\begin{document}
 
\begin{center}
	\fbox{\fbox{\parbox{5.5in}{\centering Istituti Card. C. Baronio - Vicenza \\ Anno scolastico 2018/2019 \\ Compito di Informatica \\ Idoneità 5a AFM}}}
\end{center}

\vspace{5mm}

\makebox[\textwidth]{Nome e Cognome:\enspace\hrulefill Data:\enspace\hrulefill}
 
\vspace{10mm}

Vicino ad ogni quesito, scrivi se la risposta è vera (T) o falsa (F)
\begin{multicols}{2}
	
\begin{questions} 

\question[1] \tf[T] I fogli di Excel permettono di gestire i dati aziendali

\question[1] \tf[T] Un foglio Excel è costituito da celle

\question[1] \tf[T] Posso dare un nome specifico ad ogni cella

\question[1] \tf[F] In Excel posso inserire solo valori numerici

\question[1] \tf[F] In Excel non posso scrivere su più celle contemporaneamente

\question[1] \tf[T] Per inserire una formula in Excel uso l'operatore =

\question[1] \tf[T] L'errore \#DIV/0! indica che un numero sta per essere diviso per zero

\question[1] \tf[F] Per il bloccaggio cella in Excel uso l'operatore \&

\question[1] \tf[T] La formattazione condizionale in Excel permette di colorare le celle a patto di soddisfare determinate condizioni

\question[1] \tf[F] La funzione CONTA.VALORI mi fornisce il totale delle celle che seleziono

\question[1] \tf[T] La funzione SOMMA.SE fornisce il totale delle sole celle che soddifano una condizione

\question[1] \tf[T] La tabella pivot permette di rappresentare elenchi sotto forma di tabelle o grafici riassuntivi

\question[1] \tf[F] Un database si occupa di sole parole

\question[1] \tf[F] Un database dovrebbe essere prima di tutto sicuro e scaltro

\question[1] \tf[T] L'elenco dei prezzi di telefoni è un esempio di database

\question[1] \tf[T] Un database dovrebbe essere persistente

\question[1] \tf[F] Il sistema informatico si occupa di organizzare le informazioni, il sistema informativo si occupa dei software di gestione

\question[1] \tf[T] Un'informazione è un dato inserito in un contesto

\question[1] \tf[T] Esistono 3 modelli di database: concettuali, logici e fisici

\question[1] \tf[F] Le tabelle corrispondono al livello concettuale di un database

\question[1] \tf[T] Il diagramma ER è un modello grafico di rappresentazione della realtà

\question[1] \tf[T] Un diagramma ER è indipendente dalla teconologia che viene usata per implementarlo

\question[1] \tf[F] Un attributo è una caratteristica di un collegamento tra entità

\question[1] \tf[F] La progettazione logica permette di interrogare i database

\question[1] \tf[T] Una chiave esterna permette il collegamento tra più tabelle

\end{questions}
\end{multicols}

\end{document}