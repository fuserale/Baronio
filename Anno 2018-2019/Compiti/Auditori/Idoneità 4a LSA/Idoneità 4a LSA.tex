\documentclass[addpoints]{exam}
\usepackage[utf8]{inputenc}
\usepackage{multicol}
\usepackage{graphicx}
\usepackage{amsmath}
\usepackage[a4paper,top=15mm, bottom=15mm, left=15mm, right=15mm]{geometry}

%\printanswers

\newcommand{\tf}[1][{}]{%
	\fillin[#1][0.25in]%
}
 
\begin{document}
 
\begin{center}
	\fbox{\fbox{\parbox{5.5in}{\centering Istituti Card. C. Baronio - Vicenza \\ Anno scolastico 2018/2019 \\ Compito di Informatica \\ Idoneità 4a LSA}}}
\end{center} 
 
\vspace{5mm}
 
\makebox[\textwidth]{Nome e Cognome:\enspace\hrulefill Data:\enspace\hrulefill}
 
\vspace{5mm}
 
Vicino ad ogni quesito, scrivi se la risposta è vera (T) o falsa (F)
\begin{multicols}{2}
\begin{questions}


\question \tf[T] Il linguaggio HTML permette di creare pagine WEB

\question \tf[F] Il tag body va scritto prima del tag head

\question \tf[T] HTML è un linguaggio di markup

\question \tf[T] Un tag è un comando di creazione di un elemento

\question \tf[F] Per aggiungere un link ad una pagina si usa il tag $<$link$>$

\question \tf[T] Il tag $<$image$>$ permette di inserire un'immagine

\question \tf[T] Il tag per le immagini ed il tag per i link hanno degli attributi obbligatori

\question \tf[F] In HTML non si possono creare liste

\question \tf[T] La tabella viene inserita col tag $<$table$>$

\question \tf[F] Ci sono tre liste in HTML: ordinate, non ordinate e casuali

\question \tf[F] In HTML si possono colorare solo i testi

\question \tf[T] I fogli di stile permettono di abbellire la pagina WEB

\question \tf[T] Un CMS ci permette di creare più facilmente un sito web

\question \tf[F] Il blog è l'unico tipo di pagina web che possiamo creare con i CMS

\question \tf[T] Il nuovo standard per l'HTML è la versione 5

\question \tf[F] L'elenco numerato e l'elenco puntato usano lo stesso tag

\question \tf[T] Il tag $<$title$>$ può essere inserito dove si vuole nella pagina 

\question \tf[F] Le tabelle vengono sempre create automaticamente con il bordo disegnato

\question \tf[F] Tutto il testo, in un file HTML, deve essere scritto per forza in inglese

\question \tf[T] I fogli di stile possono essere scritti in file diversi da quelli di HTML


\end{questions}
\end{multicols}

\end{document}