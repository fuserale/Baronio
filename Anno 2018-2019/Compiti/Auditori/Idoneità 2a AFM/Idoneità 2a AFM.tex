\documentclass[addpoints]{exam}
\usepackage[utf8]{inputenc}
\usepackage{multicol}
\usepackage{graphicx}
\usepackage{amsmath}
\usepackage[a4paper,top=15mm, bottom=15mm, left=15mm, right=15mm]{geometry}
 
 
%\printanswers

\newcommand{\tf}[1][{}]{%
	\fillin[#1][0.25in]%
}

\begin{document}
 
\begin{center}
	\fbox{\fbox{\parbox{5.5in}{\centering Istituti Card. C. Baronio - Vicenza \\ Anno scolastico 2018/2019 \\ Compito di Informatica \\ Idoneità 2a AFM}}}
\end{center}

\vspace{5mm}

\makebox[\textwidth]{Nome e Cognome:\enspace\hrulefill Data:\enspace\hrulefill}
 
\vspace{10mm}

Vicino ad ogni quesito, scrivi se la risposta è vera (T) o falsa (F)
\begin{multicols}{2}
	
\begin{questions} 

\question[1] \tf[T] L'informatica si occupa di studiare gli algoritmi da far eseguire ad una macchina

\question[1] \tf[F] Ci sono 9 generazioni principali di macchine informatiche

\question[1] \tf[F] Un'informazione è una descrizione elementare

\question[1] \tf[T] Un'informazione è costituita da dati che assumono significato in un certo contesto

\question[1] \tf[T] L'algebra booleana permette di combinare più affermazioni attraverso le tavole di verità

\question[1] \tf[F] Una variabile booleana può assumere qualsiasi valore

\question[1] \tf[T] L'operatore OR equivale ad un "e" nella lingua italiana

\question[1] \tf[F] L'alimentatore permette al PC di fare calcoli

\question[1] \tf[T] La RAM è una memoria ad alto rendimento ma volatile

\question[1] \tf[F] L'hardware si occupa delle applicazioni, il software delle componenti fisiche

\question[1] \tf[T] La scheda madre permette di collegare tutti i componenti di un PC

\question[1] \tf[T] La scheda video è particolarmente importante nei videogiochi

\question[1] \tf[T] Il disco di memoria può essere HDD o SSD

\question[1] \tf[F] Esistono solo software di alto livello

\question[1] \tf[F] Firmware e Sistema Operativo sono la stessa cosa

\question[1] \tf[F] Una rete di calcolatori è composta da al massimo un PC

\question[1] \tf[T] Il DNS permette di associare un servizio ad un nome invece che ad un numero

\question[1] \tf[T] L'url è il collegamento che dobbiamo inserire per accedere ad un sito

\question[1] \tf[T] La Privacy è il diritto alla riservatezza delle informazioni personali e della propria vita personale

\question[1] \tf[F] La procedura di accesso ad un sito è: autorizzazione, identificazione e autentificazione

\question[1] \tf[F] Caompito123 è una password altamente sicura

\question[1] \tf[T] La patch è la correzione di un determinato errore

\question[1] \tf[T] Uno spyware carpisce le informazioni di una persona senza che l'utente ne sia consapevole

\question[1] \tf[F] Il diritto d'autore non deve essere applicato su ogni opera originale

\question[1] \tf[F] Software Open Source significa che chiunque lo può modificare

\end{questions}
\end{multicols}

\end{document}