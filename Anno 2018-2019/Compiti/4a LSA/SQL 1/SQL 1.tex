\documentclass[addpoints]{exam}
\usepackage[utf8]{inputenc}
\usepackage{multicol}
 
\begin{document}
 
 
\vspace{5mm}
 
\makebox[\textwidth]{Nome e Cognome:\enspace\hrulefill}
 
\vspace{5mm}
 
\makebox[\textwidth]{Data e Classe:\enspace\hrulefill}
 
\begin{questions}
\question[3] Crea il diagramma ER a partire dalla seguente descrizione:\\ "Si vuole progettare un database sulle schede degli esercizi proposte dagli istruttori in una palestra. Per ogni esercizio, si salva il corrispettivo codice, il nome e una breve descrizione. Degli istruttori si vogliono sapere il codice del tesserino, il nome ed il cognome. Della scheda si conosce invece il codice di numerazione e la difficoltà. Una scheda è composta da più esercizi e viene preparata da un solo istruttore"

\question[2] A partire dal diagramma ER che hai appena creato, fornisci il corrispettivo schema relazionale

\question Esegui le seguenti operazioni SQL sullo schema relazionale che hai fornito:
\begin{parts}
	\part[1] Crea le tabelle dello schema relazionale;
	\part[1] Inserisce nella tabella ISTRUTTORE i valori "T001", "Mario", "Rossi" e nella tabella ESERCIZIO i valori "E005", "Squat", "Piegarsi con le ginocchia"
	\part[1] Visualizza tutte le schede con difficoltà "Difficile"
	\part[1] Visualizza il nome degli istruttori che hanno proposto l'esercizio con codice "E041"
	\part[1] Visualizza le difficoltà delle schede che contengono gli esercizi con nome "addominali"
\end{parts}



 
\end{questions}

\begin{center}
	\gradetable[h][questions]
\end{center}

\end{document}