\documentclass[addpoints]{exam}
\usepackage[utf8]{inputenc}
\usepackage{multicol}
 
\begin{document}
 
 
\vspace{5mm}
 
\makebox[\textwidth]{Nome e Cognome:\enspace\hrulefill}
 
\vspace{5mm}
 
\makebox[\textwidth]{Data e Classe:\enspace\hrulefill}
 
\begin{questions}
\question[3] Crea il diagramma ER a partire dalla seguente descrizione:\\ "Si vuole progettare un database sulle polizze assicurative. Un cliente può stipulare una o più polizze ed ogni polizza può assicurare uno o più beni. Di un cliente di conoscono il codice fiscale, il cognome ed il nome. La polizza viene registrata con un ID, un nome e la data di creazione della stessa. Il bene assicurato possiede un ID, un nome e la descrizione, oltre che un valore."

\question[2] A partire dal diagramma ER che hai appena creato, fornisci il corrispettivo schema relazionale

\question Esegui le seguenti operazioni SQL sullo schema relazionale che hai fornito:
\begin{parts}
	\part[1] Crea le tabelle dello schema relazionale;
	\part[1] Inserisce nella tabella CLIENTE i valori "C005", "Rossi", "Mario" e nella tabella BENE i valori "B548", "Casa", "Bene immobile"
	\part[1] Visualizza tutte le polizze create il giorno 24/12/2018
	\part[1] Visualizza il nome di tutte le polizze create da Luca Faggiano
	\part[1] Visualizza il valore di tutti i beni creati dai signori Ferrari
\end{parts}



 
\end{questions}

\begin{center}
	\gradetable[h][questions]
\end{center}

\end{document}