\documentclass[addpoints]{exam}
\usepackage[utf8]{inputenc}
\usepackage{multicol}
 
\begin{document}
 
 \begin{center}
 	\fbox{\fbox{\parbox{5.5in}{\centering
 				\huge{ARRAY E FUNZIONI C++}}}}
 \end{center}
 
\vspace{5mm}
 
\makebox[\textwidth]{Nome e Cognome:\enspace\hrulefill}
 
\vspace{5mm}
 
\makebox[\textwidth]{Data e Classe:\enspace\hrulefill}
 
\vspace{10mm}

\begin{center}
	\huge{\textbf{SEZIONE A}}
\end{center}

Nello stesso programma, svolgi i seguenti punti, commentando le varie sezioni:
\begin{questions}
	\begingradingrange{a1}
	
\question[1] Crea un array di interi di dimensione data dall'utente e riempilo con numeri dati in input

\question[1] Stampa l'array

\question[2] Calcola la media dei valori contenuti nell'array

\question[2] Crea un nuovo array in cui copi gli elementi dell'array originale che sono superiori alla media dei valori dell'array originale

\question[2] Scrivi una funzione che, preso in input un array, stampa a video quanti numeri pari e quanti numeri dispari sono presenti al suo interno

\question[2] Crea una matrice quadrata (numero colonne = numero righe) e stampala a video

	\endgradingrange{a1}

\begin{center}
	\partialgradetable{a1}[h][questions]
\end{center}

\begin{center}
	\huge{\textbf{SEZIONE B}}
\end{center}

In alternativa, sviluppa il codice a funzioni con le matrici seguendo i seguenti punti:

	\begingradingrange{a2}
	
\question[2] Funzione che carica valori casuali nella matrice

\question[2] Funzione che visualizza la matrice in maniera ordinata

\question[2] Funzione che stampa il totale di una colonna e di una riga k

\question[2] Funzione che stampa la diagonale secondaria della matrice

\question[2] Funzione che calcola la media della matrice

	\endgradingrange{a2}
\end{questions}


\begin{center}
	\partialgradetable{a2}[h][questions]
\end{center}

\end{document}