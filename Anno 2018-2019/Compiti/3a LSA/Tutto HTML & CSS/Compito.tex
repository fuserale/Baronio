\documentclass[addpoints]{exam}
\usepackage[utf8]{inputenc}
\usepackage{multicol}
 
\begin{document}
 
 
\vspace{5mm}
 
\makebox[\textwidth]{Nome e Cognome:\enspace\hrulefill}
 
\vspace{5mm}
 
\makebox[\textwidth]{Data e Classe:\enspace\hrulefill}
 
\begin{questions}
\question Rispondi alle seguenti domande:
\begin{multicols}{2}
	\begin{parts}
		\part[\half] HTML è un linguaggio di
		\begin{checkboxes}
			\choice Programmazione
			\choice Descrizione
			\choice Sperimentazione
			\choice SubProgrammazione
		\end{checkboxes}
	
		\part[\half] Il CSS serve a 
		\begin{checkboxes}
			\choice Inserire contenuti
			\choice Inserire immagini
			\choice Abbellire la pagina
			\choice Controllare la pagina
		\end{checkboxes}
	
		\part[\half] Il tag per il link è 
		\begin{checkboxes}
			\choice a
			\choice p
			\choice br
			\choice hr
		\end{checkboxes}
	
		\part[\half] Un attributo è identificato da
		\begin{checkboxes}
			\choice Coppia valore-nome
			\choice Coppia chiave-chiave
			\choice Coppia tag-valore
			\choice Coppia nome-valore
		\end{checkboxes}
	
		\part[\half] La struttura di una pagina WEB:
		\begin{checkboxes}
			\choice head, doctype, body
			\choice head, body, doctype
			\choice body, doctype, head
			\choice doctype, head, body
		\end{checkboxes}
	
		\part[\half] Il tag DIV:
		\begin{checkboxes}
			\choice E' un contenitore
			\choice Separa contenuti
			\choice Serve per i titoli
			\choice Colora il testo
		\end{checkboxes}
	
		\part[\half] Gli attributi del CSS vanno inseriti
		\begin{checkboxes}
			\choice Nel tag di apertura
			\choice Nel tag di chiusura
			\choice In entrambi i tag
			\choice All'inizio del documento
		\end{checkboxes}
	
		\part[\half] Invio dati in un form:
		\begin{checkboxes}
			\choice text
			\choice textarea
			\choice button
			\choice url
		\end{checkboxes}
	\end{parts}
\end{multicols}

\question Crea la seguente pagina web:
\begin{parts}
	\part[1] Come titolo principale metti "Google Stadia" e posizionalo al centro, di colore rosso
	\part[1] Metti un paragrafo in cui metti il seguente testo: "Google Stadia è una piattaforma di cloud gaming grazie alla quale è possibile giocare (tecnicamente a qualsiasi titolo) in streaming su qualsiasi PC e smartphone e tablet (selezionati) con installato il browser Chrome o il sistema operativo ChromeOS.", con uno sfondo blu e testo bianco.
	\part[1] Cerca nel web un'immagine di Google Stadia ed inseriscila nella pagina
	\part[1] Crea una lista con le novità principali di Google Stadia: potenza estrema, nessuna console fisica, giochi ad alto framerate, niente aggiornamenti, varietà di giochi, sicurezza
	\part[1] Crea una tabella con una comparativa dei seguenti dati: \\ Metro Exodus: Xbox One 30 FPS, Playstation 30 FPS, Google Stadia 90 FPS \\ Tomb Raider: Xbox One 25 FPS, Playstation 30 FPS, Google Stadia 80 FPS \\ Doom: Xbox One 45 FPS, Playstation 50 FPS, Google Stadia 144 FPS \\ Assassin's Creed: Xbox One 60 FPS, Playstation 60 FPS, Google Stadia 180 FPS
	\part[1] Inserisci un form per l'inserimento da parte dell'utente di: nome, cognome, mail e recensione (testo largo), più un menù di scelta del gioco preferito tra quelli elencati prima
\end{parts}



 
\end{questions}

\begin{center}
	\gradetable[h][questions]
\end{center}

\end{document}