\documentclass[addpoints]{exam}
\usepackage[utf8]{inputenc}
\usepackage{multicol}
 
\begin{document}
 
 \begin{center}
 	\fbox{\fbox{\parbox{5.5in}{\centering
 				\huge{HTML \& CSS}}}}
 \end{center}
 
\vspace{5mm}
 
\makebox[\textwidth]{Nome e Cognome:\enspace\hrulefill}
 
\vspace{5mm}
 
\makebox[\textwidth]{Data e Classe:\enspace\hrulefill}
 
\vspace{10mm}
Un negozio, chiamato Sergio \& Co. vi ha incaricato di creare il loro sito personale, in cambio di un compenso economico e di visibilità. Vi ha detto che più il sito sarà bello e funzionale, più il compenso sarà alto. Il negozio è suddiviso in più piani, al cui interno ciascuno offre degli articoli diversi. \\
Le sue specifiche per il sito sono le seguenti:
 
\begin{questions}
	
\question[1] Nome del negozio come titolo principale del negozio;

\question[4] Una sezione per la parte di cancelleria, con almeno 8 oggetti da mettere in vendita, in cui nella parte di sinistra si deve avere un'immagine di cancelleria e nella parte di destra una tabella con il nome dell'oggetto ed il prezzo;

\question[4] Una sezione per la parte di giocheria (lego, palloni, etc.), con almeno 8 oggetti da mettere in vendita, in cui nella parte di sinistra si deve avere un'immagine di giocheria e nella parte di destra una tabella con il nome dell'oggetto ed il prezzo;

\question[4] Una sezione per la parte di vestiti, con almeno 8 oggetti da mettere in vendita, in cui nella parte di sinistra si deve avere un'immagine di vestiti e nella parte di destra una tabella con il nome dell'oggetto ed il prezzo;

\question[3] Una barra di navigazione dove inserire i collegamenti veloci alle varie sezioni, ossia ai vari piani del negozio, posta in alto nella pagina, subito sotto il titolo;

\question[4] Un questionario finale (sotto forma di form) in cui il cliente può chiedere informazioni al negozio, inserendo il proprio nome, cognome, numero di telefono, la sezione a cui fa riferimento la domanda (sotto forma di opzione di scelta) e un messaggio da inviare, oltre che ad un grado di soddisfazione espresso in numero da 1 a 10;

\end{questions}

\begin{center}
	\gradetable[h][questions]
\end{center}

\end{document}