\documentclass[addpoints]{exam}
\usepackage[utf8]{inputenc}
\usepackage{multicol}
\usepackage{graphicx}
\usepackage{amsmath}
\usepackage[a4paper,top=15mm, bottom=15mm, left=15mm, right=15mm]{geometry}

\begin{document}
	
\begin{center}
	\fbox{\fbox{\parbox{5.5in}{\centering Istituti Card. C. Baronio - Vicenza \\ Anno scolastico 2018/2019 \\ Compito di Informatica \\ Recupero 1a TL}}}
\end{center}

\vspace{5mm}

\makebox[\textwidth]{Nome e Cognome:\enspace Giovanni Peruzzo Data:\enspace 26/09/2019}

\vspace{10mm}

Il candidato risponda alle seguenti 3 domande usando lo spazio fornito:\\
\begin{enumerate}
	\item Si spieghi la differenza di funzionalità tra firmware, sistema operativo e applicazione e si indichi quale di questi può essere considerato di alto livello (più vicino all’esperienza d’uso dell’utente) e quale di basso livello (che funziona senza l’intervento dell’utente) \fillwithlines{2in}
	
	\item Si definiscano due tra i seguenti concetti, fornendo anche un esempio reale: virus, worm, trojan, exploit, spyware, patch\fillwithlines{2in}
	
	\item Si spieghi cosa si intende con il concetto di privacy e si faccia un esempio reale di violazione della stessa. Si faccia un elenco, poi, di buone pratiche per la gestione e la scelta delle password personali \fillwithlines{2in}
\end{enumerate}

\end{document}