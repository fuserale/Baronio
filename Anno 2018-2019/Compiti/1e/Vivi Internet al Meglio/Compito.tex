\documentclass[addpoints]{exam}
\usepackage[utf8]{inputenc}
\usepackage{multicol}
\usepackage[a4paper,top=15mm, bottom=15mm, left=15mm, right=15mm]{geometry}
 
\begin{document}
 
 \begin{center}
 	\fbox{\fbox{\parbox{5.5in}{\centering
 				\huge{VIVI INTERNET AL MEGLIO}}}}
 \end{center}
 
\vspace{5mm}
 
\makebox[\textwidth]{Nome e Cognome:\enspace\hrulefill}
 
\vspace{5mm}
 
\makebox[\textwidth]{Data e Classe:\enspace\hrulefill}
 
\vspace{10mm}


\begin{questions}
\question Condividi usando il buon senso
\begin{parts}
	\part[10] Spiega cosa si intende per ombra digitale e perché è importante
	
	\fillwithlines{2in}
	
	\part[10] Spiega cosa si intende con violazione della privacy e fai degli esempi di violazione della privacy in internet
	
	\fillwithlines{2in}
	
	\part[5] Caso di studio: Marco vuole augurare una buona vacanza a Luca su Facebook. A cosa deve fare attenzione Marco?
	
	\fillwithlines{2in}
\end{parts}

\pagebreak

\question Distingui il falso dal vero
\begin{parts}
	\part[10] Spiega cos'è il phishing e quali sono gli strumenti principali per attuarlo
	
	\fillwithlines{2in}
	
	\part[10] A cosa bisogna fare attenzione per riconoscere i tentativi di phishing? Fai degli esempi pratici
	
	\fillwithlines{2in}
	
	\part[5] Caso di studio: considerando il dominio www.baroniovicenza.it, da quali indirizzi potrebbero arrivare tentativi di phishing? \\
	\begin{checkboxes}
		\choice assistenza@baroniovicenza.it
		\choice https://www.baronevicenza.it
		\choice http://www.baronio-vicenza.it
		\choice info@baroniovicenza.ti
		\choice www.baroniovicenza.it/login
		\choice baronio@vicenza.ru
		\choice https://www.login.baroniovicenza.it
		\choice https://www.baroniovicenza.personali.it
	\end{checkboxes}
\end{parts}

\pagebreak

\question Custodisci le tue informazioni

\begin{parts}
	\part[10] Spiega quali sono le buone abitudini per creare delle password efficaci
	
	\fillwithlines{2in}
	
	\part[10] Spiega in cosa consiste la verifica in due passaggi e quali strumenti si possono usare per attuarla
	
	\fillwithlines{2in}
	
	\part[5] Caso di studio: devi creare un progetto condiviso con i tuoi compagni di classe attraverso il tuo account, per cui devi condividere la tua password. Che accorgimenti prendi?
	
	\fillwithlines{2in}
	


\end{parts}

\pagebreak

\question Condividi la gentilezza
\begin{parts}
	
	\part[10] Spiega cosa si intende col termine cyberbullismo e quali sono i soggetti coinvolti
	
	\fillwithlines{2in}
	
	\part[10] Spiega nel dettaglio uno dei tipi presentati in classe di Cyberbullismo
	
	\fillwithlines{2in}
	
	\part[5] Caso di studio: hanno creato una pagina su di te, in cui vengono condivise le tue conversazioni, immagini e meme. Come ti comporti?\\
	\begin{checkboxes}
		\choice Crei una pagina su tutti i tuoi amici per vendicarti
		\choice Blocchi la pagina
		\choice Eviti di andare a guardare la pagina
		\choice Rispondi con gentilezza ai post e commenti che postano
		\choice Denunci la cosa alle forze dell'ordine competenti
		\choice Commenti i post con minacce
		\choice Segnali l'episodio ad un adulto di tua fiducia
		\choice Pubblichi anche tu dei post esileranti
		\choice Segnali tutte le persone che hanno interagito con la pagina
		\choice Chiedi ai tuoi amici di aiutarti nel debellare la pagina
	\end{checkboxes}
\end{parts}

\end{questions}
\begin{center}
	\gradetable[h][questions]
\end{center}



\end{document}