\documentclass[addpoints]{exam}
\usepackage[utf8]{inputenc}
\usepackage{multicol}
 
\begin{document}
 
\begin{center}
\fbox{\fbox{\parbox{5.5in}{\centering
Rispondi alle domande nello spazio predisposto.Il punteggio per ogni domanda è indicato sul lato.
Se hai domande di chiarimento, alza la mano.}}}
\end{center}
 
\vspace{5mm}
 
\makebox[\textwidth]{Nome e Cognome:\enspace\hrulefill}
 
\vspace{5mm}
 
\makebox[\textwidth]{Data e Classe:\enspace\hrulefill}
 
\begin{questions}
	
\question[2] Dai una definizione di software e spiega la differenza tra “software di alto livello” e “software di basso livello”
\fillwithlines{1in}
 
\question[3] Ordina i livelli del procollo di comunicazione TCP/IP

\begin{multicols}{2}
	\begin{checkboxes}
		\choice Rete
		\choice Fisico
		\choice Sessione
		\choice Presentazione
		\choice Applicazione
		\choice Collegamento dati
		\choice Trasporto
	\end{checkboxes}
\end{multicols}
 
\question[1] Quale tra questi è un indirizzo IP?

\begin{oneparcheckboxes}
	\choice 192.168.ABC.3
	\choice 192.168:34.1
	\choice 192.168.31.1
	\choice 192.168,31.1
\end{oneparcheckboxes}

\question[1] Quale tra questi è un indirizzo URL?

\begin{oneparcheckboxes}
	\choice protocollo://dominio.computer.estensione/file?altro
	\choice protocollo://computer.dominio.estensione/file?altro
	\choice computer://estensione.dominio.protocollo/file?altro
	\choice protocollo://computer.dominio.estensione/testata
\end{oneparcheckboxes}

\question[1] Privacy: quali tra questi dati devono essere regolamentati?

\begin{oneparcheckboxes}
	\choice Nome
	\choice Fotografia salvata sul telefono
	\choice PC posseduto
	\choice Sanzioni amministrative
	\choice Stato di salute
	\choice Post pubblicato su un social
	\choice Voto scolastico
	\choice Shampoo usato
\end{oneparcheckboxes}

\question[1] Indica il livello di sicurezza di una password assegnando i seguenti numeri: 1 per password non sicura, 2 per password sicura e 3 per password impenetrabile

\begin{oneparcheckboxes}
	\choice 123456
	\choice \$GomOrrA\#
	\choice Baronio173
	\choice IMpen\$e30tra(bile\&
	\choice 6666666
	\choice cne32432\%\$UII\=RT
	\choice SonoInutile23
	\choice \$\%\&\$\%
\end{oneparcheckboxes}

\question[2] Cos’è il diritto d’autore e su cosa deve essere applicato?
\fillwithlines{1in}

\question[3] Cerca di spiegare come funziona un motore di ricerca
\fillwithlines{2in}

\question[1] A cosa serve il firmware?

\begin{checkboxes}
	\choice Verificare i componenti del PC al suo spegnimento
	\choice Verificare i componenti del PC alla sua accensione
	\choice Dare comandi al sistema operativo
\end{checkboxes}

\question[1] Quali tra i seguenti elementi vengono gestiti dal Kernel?

\begin{oneparcheckboxes}
	\choice Processi
	\choice Hardware
	\choice Memoria Centrale
	\choice Software
	\choice File System
	\choice Internet
\end{oneparcheckboxes}	

\question[1] Ordina il processo di accesso ad un qualsiasi sistema informatico sicuro

\begin{oneparcheckboxes}
	\choice Autorizzazione
	\choice Identificazione
	\choice Autentificazione
\end{oneparcheckboxes}	

\question[3] Collega il corrispettivo attacco informatico con la caratteristica adeguata

\begin{multicols}{2}
	\begin{checkboxes}
		\choice Malware
		\choice Virus
		\choice Worm
		\choice Trojan
		\choice Vulnerabilità
		\choice Explit
		\choice Spyware
		\choice Patch
		\choice Si muove da solo nella rete
		\choice Non ha la capacità di auto-replicarsi
		\choice Corregge un determinato software
		\choice Qualsiasi minaccia informatica
		\choice Ha bisogno dell’utente per diffondersi
		\choice Ottiene le informazioni di una persona
		\choice È una debolezza in un software
		\choice Crea punti di
		accesso privilegiati
	\end{checkboxes}	
\end{multicols}
 
\end{questions}

\begin{center}
	\combinedgradetable[h][questions]
\end{center}

\end{document}