\documentclass{article}
\usepackage[utf8]{inputenc}
\usepackage[a4paper,top=15mm, bottom=15mm, left=15mm, right=15mm]{geometry}
\usepackage{multicol}

\begin{document}
	\begin{center}
		\begin{huge}
			OBIETTIVI PER LEZIONE - ELETTRONICA 4a TL
		\end{huge}
	\end{center}

	\begin{center}
		\fbox{\fbox{\parbox{5.5in}{\centering \textbf{Schema generale di lavoro}
					\begin{enumerate}
						\item Presentazione obiettivi della lezione (3 minuti)
						\item Lavoro di gruppo e studio dell'argomento (20-25 minuti)
						\item Discussione in classe dell'argomento con costruzione della mappa riassuntiva (15 minuti)
						\item Eventuali domande (tempo rimasto)
		\end{enumerate}}}}
	\end{center}

	\begin{center}
		\textbf{Lezione 1 - Magnetismo e induzione del campo magnetico (20/09)}

	\begin{enumerate}
		\item Conoscere il concetto di magnetismo naturale e di bipolo
		\item Conoscere le caratteristiche delle linee di forza
		\item Conoscere l'induzione di un campo magnetico, portando l'esempio di un filo
		\item Conoscere la formula della forza magnetomotrice
	\end{enumerate}


		\textbf{Lezione 2 - Intensità del campo magnetico e forza elettomotrice indotta (26/09)}
		\begin{enumerate}
			\item Conoscere il campo magnetico di un solenoide
			\item Conoscere il campo magnetico di un filo (legge di Biot-Savart)
			\item Conoscere i concetti di permeabilità magnetica, induzione e flusso magnetico
			\item Conoscere la legge di Faraday-Neumann-Lenz e come si calcola la forza elettromotrice indotta in un conduttore
		\end{enumerate}

		\textbf{Lezione 3 - Autoinduzione e induttanza (26/09)}
		\begin{enumerate}
			\item Conoscere il concetto e la formula dell'induttanza
			\item Conoscere quando è utile o meno usare il fenomeno dell'autoinduzione
			\item Comprendere cosa succede quando si apre o chiude un circuito induttivo
			\item Comprendere cosa succede all'energia magnetica all'apertura e chiusura di un circuito induttivo
		\end{enumerate}

		\textbf{Lezione 4 - Esercitazione magnetismo (01/10)}
		\begin{enumerate}
			\item Saper risolvere esercizi sul campo magnetico
			\item Saper risolvere esercizi sul flusso del campo magnetico
			\item Saper risolvere esercizi sulla forza elettromotrice
		\end{enumerate}

		\textbf{Lezione 5 - Mutua induzione tra circuiti \& Forza elettromagnetica (01/10)}
		\begin{enumerate}
			\item Conoscere il concetto di mutua induzione e perché si viene a creare
			\item Sapere la formula del coefficiente di mutua induzione e della fem indotta
			\item Sapere dove ha notevole importanza il fenomeno di mutua induzione, portando degli esempi di macchine che lo usano
			\item Conoscere le formule di mutua induzione per circuiti accoppiati
			\item Sapere come si crea la forza elettromagnetica e come si calcola
			\item Conoscere il funzionamento di un elettromagnete
		\end{enumerate}

		\textbf{Lezione 6 - Materiali paramagnetici, diamagnetici e ferromagnetici (04/10)}
		\begin{enumerate}
			\item Conoscere la formula del campo magnetico in presenza di un materiale
			\item Sapere la differenza tra materiali paramagnetici, diamagnetici e ferromagnetici
			\item Conoscere il comportamento dei materiali ferromagnetici (curva di prima magnetizzazione) ed il concetto di punto di Curie
			\item Saper leggere il ciclo di isteresi di un materiale ferromagnetico
		\end{enumerate}

	\textbf{Lezione 7 - Riassunto capitolo 1 (08/10)}
	\begin{enumerate}
	\item Rispondere alle domande proposte
	\item Risolvere gli esercizi proposti
	\end{enumerate}

	\textbf{Lezione 8 - Rappresentazioni sinusoidali (08/10)}
	\begin{enumerate}
	\item Conoscere perché sono importanti i segnali sinusoidali nell'elettrotecnica
	\item Conoscere e saper usare la rappresentazione vettoriale
	\item Conoscere e saper usare la rappresentazione complessa
	\item Conoscere e saper usare la rappresentazione esponenziale
	\item Saper passare da una rappresentazione all'altra
	\end{enumerate}

	\textbf{Lezione 9 - Impedenza (11/10)}
	\begin{enumerate}
	\item Conoscere la definizione di impedenza
	\item Sapere la formula dell'impedenza e della reattanza per l'induttore
	\item Sapere la formula dell'impedenza e della reattanza per il condensatore
	\item Sapere la formula dell'impedenza e della reattanza per la resistenza
	\end{enumerate}

	\textbf{Lezione 10 - Esercitazione impedenza}
	\begin{enumerate}
	\item Riuscire a risolvere gli esercizi sull'impedenza dei circuiti
	\end{enumerate}

	\textbf{Lezione 11 - Risonanza in serie e in parallelo}
	\begin{enumerate}
	\item Conoscere i principi della risonanza in serie e come si calcola
	\item Conoscere i principi della risonanza in parallelo e come si calcola
	\item Conoscere il comportamento dei condensatori e induttori reali
	\end{enumerate}
	
%	\textbf{Lezione 12 - Metodo simbolico}
%	\begin{enumerate}
%	\item Comprendere il metodo simbolico
%	\item Risolvere l'esercizio usando il metodo simbolico
%	\end{enumerate}
	
	\end{center}
\end{document}