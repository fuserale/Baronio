\documentclass[addpoints]{exam}
\usepackage[utf8]{inputenc}
\usepackage{multicol}
\usepackage{graphicx}
\usepackage{amsmath}
\usepackage{subcaption}
\usepackage[a4paper,top=15mm, bottom=15mm, left=15mm, right=15mm]{geometry}

%\printanswers

\newcommand{\tf}[1][{}]{%
	\fillin[#1][0.25in]%
}
 
\begin{document}
 
\begin{center}
	\fbox{\fbox{\parbox{5.5in}{\centering Istituti Card. C. Baronio - Vicenza \\ Anno scolastico 2019/2020 \\ Compito di Informatica - 1a TL}}}
\end{center}

\vspace{5mm}

\makebox[\textwidth]{Nome e Cognome:\enspace\hrulefill Data:\enspace\hrulefill}
 
\vspace{10mm}



\begin{questions}

\question[2] Spiega, aiutandoti casomai con un disegno, il modello di Von Neumann, esplicitando poi il ciclo del processore Fetch-Decode-Execute

\question[1] Spiega come si passa dal problema al programma, aiutandoti casomai con uno schema

\question[1]
\begin{multicols}{2}
1 B \\
1 MB \\
1 GB \\
1 TB \\\\
....... bit \\
....... byte \\
....... byte \\
....... byte \\
\end{multicols}

\question[2] Fai i collegamenti giusti
\begin{multicols}{2}
1.Supercomputer \\ 2.NoW e CoW \\ 3.Server \\ 4.Workstation \\ 5.Embedded \\ 6.Monouso \\ 7.Robotica avanzata \\ 8.Robotica industriale \\\\
A.Ha elevate prestazioni \\ B.Usati per simulazioni mediche \\ C.Li trovi nei vestiti \\  D.Raccolgono i dati degli utenti \\  E.Sono pensati per una singola applicazione \\ F.Ogni calcolo viene scomposto in sottocalcoli \\ G.Tratta robot in ambienti strutturati \\ H.Tratta robot in ambienti non strutturati
\end{multicols}

\question[1] Spiega la differenza tra licenze OEM e licenze ESD

\question[2] Spiega le 4 tipologie principali di malware: virus, worm, trojan, spyware

\question[1] Fai i collegamenti giusti
\begin{multicols}{2}
A. Kernel \\ B.Gestione della memoria \\ C.Gestione delle periferiche \\ D.File System \\ D. CLI e GUI \\\\
1. Usa code o buffer \\ 2. Interfaccia utente e terminale \\ 3. Gestisce la CPU \\ 4. Fraziona i programmi in segmenti \\ 5. Gestisce le informazioni
\end{multicols}


\end{questions}


\begin{center}
	\gradetable[h][questions]
\end{center}




\end{document}