\documentclass{article}
\usepackage[utf8]{inputenc}
\usepackage[a4paper,top=15mm, bottom=15mm, left=15mm, right=15mm]{geometry}
\usepackage{multicol}
\usepackage{hyperref}

\begin{document}
	\begin{center}
		\begin{huge}
			OBIETTIVI PER LEZIONE - INFORMATICA 2a AFM
		\end{huge}
	\end{center}

	\begin{center}
		\fbox{\fbox{\parbox{5.5in}{\centering \textbf{Schema generale di lavoro}
					\begin{enumerate}
						\item Presentazione obiettivi della lezione (3 minuti)
						\item Lavoro di gruppo e studio dell'argomento (20-25 minuti)
						\item Discussione in classe dell'argomento con costruzione della mappa riassuntiva (15 minuti)
						\item Eventuali domande (tempo rimasto)
		\end{enumerate}}}}
	\end{center}

	\begin{center}
		\textbf{Lezione 1 - La posta elettronica (23/09)}
		\begin{enumerate}
			\item Comprendere il significato di IMAP e di conseguenza quelli di web mail e POP mail, definendo i 3 attori principali della gestione della posta elettronica
			\item Comprendere la differenza tra spam e phishing
			\item Conoscere gli elementi principali dell'account di posta
			\item Fare l'esercizio 2 a pag. 106, dove invii la mail al professore, in cc metti i componenti del tuo gruppo ed in ccn gli altri tuoi compagni
		\end{enumerate}
		\textbf{Lezione 2 - Ricerche sul web (23/09)}
		\begin{enumerate}
			\item Comprendere le 4 funzioni principali di un motore di ricerca
			\item Comprendere come funziona uno spider o crawler
			\item Conoscere il concetto di indicizzazione ed a cosa serve
			\item Conoscere i criteri per valutare i risultati di ricerca
			\item Conoscere il Deep Web ed il Dark Web
			\item Comprendere i comandi filetype, define e site
		\end{enumerate}
		\textbf{Lezione 3- Ricerche sul web (03/10)}
		\begin{enumerate}
			\item Esercizio 4 pagina 119
		\end{enumerate}
	
		\begin{center}
			\textbf{Lezione 4 - Immagini e suoni digitali (03/10)}
			\begin{enumerate}
				\item Conoscere cosa si intende per immagine raster ed immagine vettoriale
				\item Sapere gli elementi caratterizzanti di un pixel
				\item Saper calcolare i bit necessari per un'immagine
				\item Conoscere come viene digitalizzato un'immagine analogica
				\item Conoscere il significato di aspect ratio 
				\item Sapere le differenze tra RGB e CMY
				\item Conoscere come si passa da un suono analogico ad uno digitale
				\item Conoscere le caratteristiche principali dei suoni digitali
			\end{enumerate}
		\end{center}
	\end{center}
	
	\begin{center}
	\textbf{Lezione 5 - Esercitarsi con GIMP e Audacity}
	\begin{enumerate}
	\item Seguire tutorial per GIMP: https://www.gimp.org/tutorials/
	\item Seguire tutorial per Audacity:  https://manual.audacityteam.org/
	\end{enumerate}
	\end{center}

\end{document}