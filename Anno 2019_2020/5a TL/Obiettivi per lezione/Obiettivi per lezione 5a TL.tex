\documentclass{article}
\usepackage[utf8]{inputenc}
\usepackage[a4paper,top=15mm, bottom=15mm, left=15mm, right=15mm]{geometry}
\usepackage{multicol}

\begin{document}
	\begin{center}
		\begin{huge}
			OBIETTIVI PER LEZIONE - ELETTRONICA 5a TL
		\end{huge}
	\end{center}

	\begin{center}
		\fbox{\fbox{\parbox{5.5in}{\centering \textbf{Schema generale di lavoro}
					\begin{enumerate}
						\item Presentazione obiettivi della lezione (3 minuti)
						\item Lavoro di gruppo e studio dell'argomento (20-25 minuti)
						\item Discussione in classe dell'argomento con costruzione della mappa riassuntiva (15 minuti)
						\item Eventuali domande (tempo rimasto)
		\end{enumerate}}}}
	\end{center}
	
	

	\begin{center}
		\textbf{Lezione 1 - Fenomeni oscillatori (20/09)}
		\begin{itemize}
			\item Conoscere i 3 tipi principali di oscillazione
			\item Conoscere gli elementi caratteristici di un'oscillazione
			\item Conoscere i fenomeni che influenzano le onde elettromagnetiche
			\item Avere un'idea generale della classificazione delle onde elettromagnetiche
		\end{itemize}
	\end{center}

	\begin{center}
		\textbf{Lezione 2 - Propagazione delle onde elettromagnetiche (20/09)}
		\begin{itemize}
			\item Conoscere le tipologie di propagazione delle onde elettromagnetiche
			\item Conoscere le differenti onde terrestri
			\item Conoscere le caratteristiche di propagazione di un'onda elettromagnetica
		\end{itemize}
	\end{center}

	\begin{center}
		\textbf{Lezione 3 - Antenne e loro caratteristiche (27/09)}
		\begin{itemize}
			\item Conoscere gli elementi di cui bisogna tener conto in un'antenna
			\item Comprendere come funziona il dipolo a mezza onda
			\item Capire il diagramma di radiazione
			\item Comprendere la selettività e la direttività di un'antenna
		\end{itemize}
	\end{center}

	\begin{center}
		\textbf{Lezione 4 - 3 tipi di antenne (03/10)}
		\begin{itemize}
			\item Dipoli hertziani: costruzione, schema elettrico e comportamento
			\item Antenne direttive: costruzione e comportamento (con esempi di utilizzo)
			\item Antenna a telaio: costruzione e comportamento
		\end{itemize}
	\end{center}

	\begin{center}
		\textbf{Lezione 5 - La radiotrasmissione (03/10)}
		\begin{itemize}
			\item Comprendere il processo di radiocomunicazione (figura 15.36)
			\item Conoscere il processo che avviene in un'antenna trasmettente ed in una ricevente
			\item Saper distinguere tra modulazione di ampiezza e modulazione di frequenza
			\item Conoscere la struttura di un radiotrasmettitore ed il compito principale di ogni elemento
			\item Conoscere come funziona la rivelazione nel radioricevitore e come si riesce a filtrare il segnale tra tutti quelli disponibili
		\end{itemize}
	\end{center}

	\begin{center}
		\textbf{Lezione 6 - Multiplexing}
		\begin{itemize}
			\item Sapere a cosa serve un multiplexing
			\item Conoscere come funziona il multiplexing per divisione di frequenza
			\item Conoscere come funziona il multiplexing per divisione di tempo
		\end{itemize}
	\end{center}
\end{document}