\documentclass{article}
\usepackage[utf8]{inputenc}
\usepackage[a4paper,top=15mm, bottom=15mm, left=15mm, right=15mm]{geometry}
\usepackage{multicol}

\begin{document}
	\begin{center}
		\fbox{\fbox{\parbox{5.5in}{\centering \textbf{Riassunto sondaggi 2019/2020}}}}
	\end{center}
	
	\begin{center}
	\textbf{4 AFM}
	\end{center}
	\begin{itemize}
	\item \textbf{Lezione}: deve essere svolta in modo interattivo e divertente, magari anche all'aria aperta, con il professore che spiega e poi pone domande o fa schemi riassunti
	\item \textbf{Argomenti}: siti web e deep/dark web, come funzionano i principali e-commerce, l'hardware ed il montaggio di un computer, programmare, hacking e dati sensibili
	\item \textbf{Scuola}: LIM, computer con TV, assemble d'istituto, area fumatori ben delineata
	\item \textbf{Valutazione}: sondaggi sull'argomento, lavoro svolto in classe, collaborazione, relazioni con gli altri
	\item \textbf{Professore}: deve considerare le persone, essere divertente e coinvolgente, avere polso
	\end{itemize}
	
	\begin{center}
	\textbf{2 AFM}
	\end{center}
	\begin{itemize}
	\item \textbf{Lezione}: non dovrebbe essere fatta, con calma e ordinata, con domande che facciano riflettere sulle cose importanti, tanta interazione, col computer giocando
	\item \textbf{Argomenti}: photoshop, segreti dell'informatica, hackerare e costruire un computer
	\item \textbf{Scuola}: sapone nei bagni, più ragazze, TV e LIM
	\item \textbf{Valutazione}: simpatia, comportamento, lavoro in classe e impegno
	\item \textbf{Professore}: comprensivo, tranquillo, divertente e rispettoso
	\end{itemize}
	
	\begin{center}
	\textbf{5 TL}
	\end{center}
	\begin{itemize}
	\item \textbf{Lezione}: lavoro di gruppo e partecipazione generale della classe, con autonomia e poi spiegazione di eventuali parti incomprese
	\item \textbf{Argomenti}: come funziona l'elettricità per sfruttarla al meglio, elettronica dell'aereo
	\item \textbf{Scuola}: aula studio con biblioteca e sala giochi, assemble d'istituto e LIM
	\item \textbf{Valutazione}: attenzione, partecipazione, comportamento tenuto in classe
	\item \textbf{Professore}: solidale e che aiuti tutti, mentalità giovanile, innovativo, con un bel rapporto con gli alunni, competente ma che sappia ammettere i propri errori, coinvolgente
	\end{itemize}
	
	\begin{center}
	\textbf{4 TL}
	\end{center}
	\begin{itemize}
	\item \textbf{Lezione}: spiegazione del docente con intrattenimento ed esempi, fare cose pratiche che abbiano un'utilità, eventualmente all'aria aperta, con una certa autonomia iniziale
	\item \textbf{Argomenti}: collegamenti elettrici, elettronica aereo, motore elettrico, 
	\item \textbf{Scuola}: assemble d'istituto, finestre più consistenti, laboratori funzionanti, orologio nelle aule
	\item \textbf{Valutazione}: impegno ed idee che si hanno, partecipazione ed intelligenza, in base al lavoro che si fa in classe
	\item \textbf{Professore}: intrattenitore con lato comico, sapendo rendere facile ciò che è difficile
	\end{itemize}
	
	\begin{center}
	\textbf{3 TL}
	\end{center}
	\begin{itemize}
	\item \textbf{Lezione}: più pratica, magari con videolezioni, con il docente che spiega
	\item \textbf{Argomenti}: come collegare cavi
	\item \textbf{Scuola}: schermo in classe, infissi nuovi, laboratori decenti
	\item \textbf{Valutazione}: impegno e partecipazione in classe, o ancora meglio non avere assolutamente valutazioni
	\item \textbf{Professore}: simpatico e che abbia riguardo per quelli in difficoltà, paziene e più simile a noi
	\end{itemize}
	\begin{center}
	\newpage
	\textbf{1 TL}
	\end{center}
	\begin{itemize}
	\item \textbf{Lezione}: leggera con delle pause, con il docente che spiega coinvolgendo ed usando esempi e schemi
	\item \textbf{Argomenti}: creare app
	\item \textbf{Scuola}: laboratori più grandi, LIM
	\item \textbf{Valutazione}: sempre in modo positivo, in base a quello che si fa a scuola e quanto si sta attenti, ma sempre con i voti
	\item \textbf{Professore}: bravo a spiegare e coinvolgente, non rigido che dà una mano a recuperare
	\end{itemize}
	
	\begin{center}
	\textbf{4 LSA}
	\end{center}
	\begin{itemize}
	\item \textbf{Lezione}: in modo divertente, con meno libri e più slides, la spiegazione con esempi e la pratica in autonomia ma seguiti, con tanta interazione
	\item \textbf{Argomenti}: siti web, C++, hackerare
	\item \textbf{Scuola}: pizzette e campo da basket, biblioteca, schermi più grandi e nuovi, LIM
	\item \textbf{Valutazione}: in base al metodo di studio, impegno, partecipazione, secondo le abilità che dimostro in classe ed alla creatività
	\item \textbf{Professore}: simpatico, intraprendente, aperto di mente, coinvolgente e bravo a spiegare
	\end{itemize}
	
	\begin{center}
	\textbf{5 LSA}
	\end{center}
	\begin{itemize}
	\item \textbf{Lezione}: il docente che spiega (se lo fa bene), altrimenti autonomia e poi discussione, poi per la pratica esercizi fatti in autonomia dagli studenti, ma con interazione del docente, portando pochi concetti teorici alla volta
	\item \textbf{Argomenti}: bitcoin, risolvere semplici problemi del computer, montare un PC, progettare sistemi di sicurezza
	\item \textbf{Scuola}: macchinette (bar costa troppo), area relax, campo da tennis, LIM, sapone nei bagni, biblioteca per studiare nel pomeriggio
	\item \textbf{Valutazione}: attenzione, impegno, interventi
	\item \textbf{Professore}: disponibile, competente, un punto di riferimento, accondiscendente, che abbia passione, coinvolgente e che aiuta i più bisognosi
	\end{itemize}
\end{document}