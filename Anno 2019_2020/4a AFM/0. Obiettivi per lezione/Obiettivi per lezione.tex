\documentclass{article}
\usepackage[utf8]{inputenc}
\usepackage[a4paper,top=15mm, bottom=15mm, left=15mm, right=15mm]{geometry}
\usepackage{multicol}

\begin{document}
\begin{center}
\begin{huge}
OBIETTIVI PER LEZIONE - ELETTRONICA 4a AFM
\end{huge}
\end{center}

	\begin{center}
	\fbox{\fbox{\parbox{5.5in}{\centering \textbf{Schema generale di lavoro}
		\begin{enumerate}
			\item Presentazione obiettivi della lezione (3 minuti)
			\item Lavoro di gruppo e studio dell'argomento (20-25 minuti)
			\item Discussione in classe dell'argomento con costruzione della mappa riassuntiva (15 minuti)
			\item Eventuali domande (tempo rimasto)
	\end{enumerate}}}}
	\end{center}

	\begin{center}
	\textbf{Lezione 1 - Basi di dati (03/10)}
	\begin{enumerate}
	\item Conoscere la definizione di base di dati
	\item Sapere dove vengono usati e perché sono importanti
	\item Conoscere le caratteristiche principali di un database
	\end{enumerate}
	\end{center}
	
	\begin{center}
	\textbf{Lezione 2 - Sistema informativo vs Sistema informatico}
	\begin{enumerate}
	\item Conoscere quali sono i compiti di un sistema informativo
	\item Presentare un esempio di sistema informativo
	\item Conoscere quali sono i compiti di un sistema informatico
	\item Saper spiegare la differenza tra sistema informativo e sistema informatico
	\end{enumerate}
	\end{center}
	
%	\begin{center}
%	\textbf{Lezione 3 - Schemi e istanze}
%	\begin{enumerate}
%	\item Conoscere la differenza tra dato e informazione, portando un esempio pratico
%	\item Conoscere i concetti di schema, istanza, valore e attributo, facendo un disegno
%	\item Sapere la definizione di modello di dati, avendone presenti le due tipologie principali
%	\end{enumerate}
%	\end{center}


\end{document}